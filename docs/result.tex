\documentclass[a4paper]{article}

\usepackage[russian]{babel}
\usepackage[utf8x]{inputenc}
\usepackage{graphics}

\author{Мартьянов Сергей}

\title{Отчет по практике}

\begin{document}

\maketitle

\section{Постановка задачи}
Целью летней практики ФМЛ №30 в 2016 году по направлению "Web-программирование" являлось изучение принципов взаимодействия сервера и клиента в сети Интернет и создание интерактивной таблицы умножения, для использования которой необходима авторизация.

\section{Способы решений}
\begin{itemize}
\item \textbf{Использование статических страниц}

Единственным достоинством данного способа является простота реализации. Но против этого сомнительного достоинства есть уйма недостатков: невозможность отделение данных от представления, сложность обновления содержимого сайта и взаимодействия с пользователем и т.д.
\item \textbf{Использование CGI скриптов}

Этот способ позволяет обрабатывать данные на стороне сервера и отправлять их клиенту, при этом за счет использования низкоуровневых языков программирования (например Си) скорость этой обработки выше, чем скорость обработки данных средствами интерпритируемых языков программирования.

Но использование этого способа требует много ресурсов, так что при большом количестве подключений к серверу, его работоспособность может нарушисться.
\item \textbf{Использование интерпритируемых языков программирования}

При использовании этого способа, нагрузка на сервер будет меньше, чем при использовании CGI-скриптов, но и обработка донных будет проходить медленее. На мой взгляд, основным преймуществом этого способа является то, что интерпритируемые языки программирования предоставляют много встроенных функций для взаимодействия сервера и клиянта.

Поэтому данный способ и был выбран для решения поставленной задачи. На сервере использовался язык PHP, хранения данных была выбрана СУБД PostgreSQL.
\end{itemize}

\pagebreak

\section{Описание решения}
При использовании языка PHP в связке с PostgreSQL необходимо реализовать четкое разделение данных от их обработки и представления.

\section{Вывод}
За время прохождения практики я получил бесценный опыт работы в области разработки web-приложений. Особенно полезным мне показалась разработка шаблона сайта,т.к. это позваляет отделить данные от их представления, тем самым существенно облегчая дальнейшую разработку и поддержку сайта.

\end{document}